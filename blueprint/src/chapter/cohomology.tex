\chapter{Magma cohomology}\label{cohomology-chapter}

Group cohomology is a theory that constructs certain abelian groups $H^n(G,M)$ to a group $G$ acting as a module on an abelian group $M$, via a chain complex
$$ 0 \to C^0(G,M) \stackrel{d}{\to} C^1(G,M) \stackrel{d}{\to} C^2(G,M) \stackrel{d}{\to} C^3(G,M) \stackrel{d}{\to} \dots$$
where $C^n(G,M)$ is the space of functions $f: G^n \to M$, and the coboundary maps $d: C^n(G,M) \to C^{n+1}(G,M)$ are explicit maps obeying the relation $d^2=0$.  The cohomology group $H^2(G,M)$ is of particular relevance in constructing extensions of the group $G$ by $M$.

It turns out that part of this formalism can be extended to more general magmas $G$ that obey a different law than the associative law.

Let $G$ be a magma obeying some equation $E$ of the form $w_{E,1}(x_1,\dots,x_n) = w_{E,2}(x_1,\dots,x_n)$.  We let $M = (M,+)$ be an abelian group that also has a linear magma operation
$$ s \op t = as + bt$$
obeying $E$ for some $a,b \in \mathrm{End}(M)$ in the endomorphism ring of $M$.    Note that in such a linear magma, any word $w(s_1,\dots,s_n)$ takes the form
$$ w(s_1,\dots,s_n) = \sum_{i=1}^n P_{w,i}(a,b) s_i$$
for some (noncommutative) polynomials $P_{w,i}(a,b)$ in $a,b$ with natural number coefficients.  For instance,
$$ s \op s = (a+b) s$$
$$ t \op (s \op s) = (ba+b^2) s + at$$
$$ (t \op (s \op s)) \op t = (aba+ab^2) s + (a^2+b)t$$
$$ t \op ((t \op (s \op s)) \op t) = (baba+bab^2) s + (ba^2+b^2+a)t.$$
The law $E$ is then obeyed when
$$ P_{w_{E,1},i}(a,b)= P_{w_{E,2},i}(a,b)$$
for $i=1,\dots,n$.  For instance, the law 1110,
\begin{equation}\label{1110}
  x = y \op ((y \op (x \op x)) \op y)
\end{equation}
would be obeyed if one has
\begin{equation}\label{baba}
 baba+bab^2 = 1; \quad ba^2+b^2+a = 0.
\end{equation}
A solution here would be provided by $(a,b) = (\phi,-1)$, where $\phi$ solves the golden ratio equation $\phi^2 = \phi + 1$.

We consider \emph{extensions} of $G$ by $M$, which are magmas with carrier $G \times M$ with an operation of the form
\begin{equation}\label{extension}
  (x,s) \op (y,t) = (x \op y, as + bt + f(x,y))
\end{equation}
for some function $f: G \times G \to M$.  An easy induction then shows that for any word $w(x_1,\dots,w_n)$, one has
$$ w((x_1,s_1),\dots,(x_n,s_n)) = (w(x_1,\dots,x_n), \sum_{i=1}^n P_{w,i}(a,b) x_i$$
$$ + \sum_{w_1 \op w_2 \leq w} Q_{w,w_1 \op w_2}(a,b) f(w_1(x_1,\dots,x_n), w_2(x_1,\dots,x_n)))$$
where $w_1 \op w_2$ ranges over all subterms of $w$ that are not single variables, and $Q_{w,w_1 \op w_2}(a,b)$ is a suitable (noncommutative) monomial in $a,b$.  For instance
$$ (x,s) \op (x,s) = (x \op x, (a+b) s + f(x,x))$$
$$ (y,t) \op ((x,s) \op (x,s)) = (y \op (x \op x), (ba+b^2) s + at + b f(x,x) + f(y, x \op x))$$
$$ ((y,t) \op ((x,s) \op (x,s))) \op (y \op t) = ((y \op (x \op x)) \op y,$$
$$ (aba+ab^2) s + (a^2+b)t + ab f(x,x) + a f(y, x \op x) + f(y \op (x \op x), y))$$
$$ (y \op t) \op (((y,t) \op ((x,s) \op (x,s))) \op (y \op t)) = (y \op ((y \op (x \op x)) \op y), $$
$$(baba+bab^2) s + (ba^2+b^2+a)t + bab f(x,x) + ba f(y, x \op x) + bf(y \op (x \op x), y) + f( y, (y \op (x \op x)) \op y)).$$
Assuming \Cref{baba}, we conclude that this extension obeys the law \Cref{1110} if and only if one has
\begin{equation}\label{cocycle-1110}
  bab f(x,x) + ba f(y, x \op x) + bf(y \op (x \op x), y) + f( y, (y \op (x \op x)) \op y) = 0
\end{equation}
for all $x,y \in G$.  More generally, an extension obeys a law $E$ provided that
$$ \sum_{w_1 \op w_2 \leq w_{E,1}} Q_{w_{E,1},w_1 \op w_2}(a,b) f(w_1(x_1,\dots,x_n), w_2(x_1,\dots,x_n))$$
$$= \sum_{w_1 \op w_2 \leq w_{E,2}} Q_{w_{E,2},w_1 \op w_2}(a,b) f(w_1(x_1,\dots,x_n), w_2(x_1,\dots,x_n))$$
for all $x_1,\dots,x_n \in G$.  We call $f$ a \emph{$E$-cocycle} if this equation holds, and denote the space of such $E$-cocycles as $Z^2_E(G,M)$.  This is an abelian group, and each $E$-cocycle defines a magma on $G \times M$ obeying $E$.  For instance, when $f=0$ we obtain the direct product of the $G$ and $M$ magmas.

Given any function $g: G \to M$, one can define a bijection $(x,s) \mapsto (x,s+g(x))$ on $G \times M$, which conjugates the law \Cref{extension} to the law
$$ (x,s) \op (y,t) = (x \op y, as + bt + f(x,y) + g(x \op y) - ag(x) - bg(y)).$$
Being conjugate, this new operation will obey $E$ if and only if the original operation does.
Thus if one defines a \emph{coboundary} to be a function $f: G \times G \to M$ of the form $f(x,y) =g(x \op y) - ag(x) - bg(y)$ for some $g: G \to M$, we can add a coboundary to an $E$-cocycle and still obtain a $E$-cocycle.  So if we let $B^2(G,M)$ be the space of coboundaries, we see that $B^2(G,M)$ is a subgroup of $Z^2_E(G,M)$.  We define the \emph{$E$-cohomology} $H^2_E(G,M)$ to be the quotient
$$ H^2_E(G,M) := Z^2_E(G,M) / B^2(G,M).$$
Observe that if $E$ implies $E'$, then $H^2_E(G,M)$ is a subgroup of $H^2_{E'}(G,M)$.  Thus, to refute an implication $E \implies E'$, it suffices to locate a magma $G$ and a linear magma $M$ obeying both $E$ and $E'$ such that
\begin{equation}\label{hgm}
  H^2_E(G,M) \not\subset H^2_{E'}(G,M).
\end{equation}

This leads to a computational approach to refutations, as these groups can be computed by linear algebra.

For instance, let us consider the law \Cref{1110} together with a putative consequence, equation 1629:
\begin{equation}\label{1629}
  x = (x \op x) \op ((x \op x) \op x).
\end{equation}
A simultaneous (linear) model for both \Cref{1110} and \Cref{1629} is given by carrier $G=M={\mathbb F}_5$ with $x \op y = 3x-y$.  Then the coboundaries $f: {\mathbb F}_5 \times {\mathbb F}_5 \to {\mathbb F}_5$ are of the form $f(x,y) = g(3x-y) - 3g(x) + g(y)$ for $g: {\mathbb F}_5 \to {\mathbb F}_5$, the $1110$-cocycles solve the equation
$$3 f(x,x) - 3 f(y, 2x) - f(3y - 2x, y) + f( y, 3y - x) = 0$$
for $x,y \in {\mathbb F}_5$, and the $1629$-cocycles solve the equation
$$ f(2x,0) - f(2x,x) = 0.$$
A function $g:{\mathbb F}_5 \to {\mathbb F}_5$ has vanishing derivative, $g(3x-y) - 3g(x) + g(y) = 0$, if and only if $g$ is linear, which is a one-dimensional space; so, by the rank-nullity theorem, the space $B^2(G,M)$ of coboundaries is four-dimensional.  One can check computationally that the space $Z^2_{1110}(G,M)$ is six-dimensional, so $H^2_{1110}(G,M)$ is two-dimensional.  One can check numerically that it contains an element not in $H^2_{1629}(G,M)$, leading to a finite counterexample on the 25-element carrier $G \times M$ to the implication of 1629 from 1110.

\begin{remark} In the case the associative law (and taking $a=b=1$), the cocycle law becomes the familiar
  $$ f(x,y) + f(x \op y, z) = f(x, y \op z) + f(y,z)$$
and we recover the usual group cohomology (if $G$ is a group).  The same occurs for Tarski's law 543
$x = y \op (z \op (x \op (y \op z)))$.
\end{remark}

\begin{remark}  One can interpret the above cohomology group in terms of a partial chain complex
$$ 0 \to C^0(G,M) \stackrel{d^0}{\to} C^1(G,M) \stackrel{d^1}{\to} C^2(G,M) \stackrel{d^2_E}{\to} C^n(G,M)$$
where the zeroth coboundary map $d^0: C^0(G,M) \to C^1(G,M)$ is the zero map, the first coboundary map $d: C^1(G,M) \to C^2(G,M)$ (which does not depend on the equation $E$) is defined by the formula
$$ d^1 vf(x,y) := f(x \op y) - (f(x) \op f(y)) = f(x \op y) - af(x) - bf(y)$$
and the second coboundary map $d^2_E: C^2(G,M) \to C^n(G,M)$ (which does depend on $E$) is defined by the formula
$$ d^2_E f(x_1,\dots,x_n) := \sum_{w_1 \op w_2 \leq w_{E,1}} Q_{w_{E,1},w_1 \op w_2}(a,b) f(w_1(x_1,\dots,x_n), w_2(x_1,\dots,x_n))$$
$$ - \sum_{w_1 \op w_2 \leq w_{E,2}} Q_{w_{E,2},w_1 \op w_2}(a,b) f(w_1(x_1,\dots,x_n), w_2(x_1,\dots,x_n)).$$
The fact that coboundaries are cocycles can then be rewritten as the chain complex relations $d^1 d^0=0$, $d^2_E d^1 = 0$.  The group $H^2_E(G,M)$ is then just the second cohomology group of this chain complex.  The first cohomology group $H^1_E(G,M)$ is the kernel of $d^1$, or equivalently the abelian group of magma homomorphisms from the magma $G$ to the linear magma $M$.

When $E$ is the associative law, this partial chain complex can be extended to the usual group cohomology chain complex; however, it is not clear if any such extension exists for a general law $E$.
\end{remark}

\begin{remark}  Suppose that $G = M = \mathbb{F}_p$ is a field of prime order, and $a,b$ are coefficients in that field, with the magma operation on $G$ also given in a linear form $x \op y = a'x  + b' y$.  Then one can view a cocycle $f: \mathbb{F}_p \times \mathbb{F}_p \to \mathbb{F}_p$ as a bivariate polynomial of degree at most $2p-2$ with coefficients in $ \mathbb{F}_p$.  The coboundary maps $d, d_E$ preserve degree, and so one can decompose (or ``grade'') the cohomology group $H^2_E(G,M)$ as $\bigoplus_{d=0}^{2p-2} H^2_E(G,M)_d$, where $H^2_E(G,M)_d$ are defined as with $H^2_E(G,M)$ but with the cocycles and coboundaries required to be homogeneous polynomials of degree at most $d$.  To disprove an implication $E \implies E'$, it thus suffices to establish a non-inclusion $H^2_E(G,M)_d \not\subset H^2_{E'}(G,M)_d$ at a single degree $d$, which may be slightly easier computationally, and also provides for more compactly described counterexamples.
\end{remark}
