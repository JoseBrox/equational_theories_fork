\chapter{Equation 1323}\label{1323-chapter}

In this chapter we study magmas that obey equation 1323,
\begin{equation}\label{1323}
  x = y \op (((y \op y) \op x) \op y)
\end{equation}
for all $x,y$.  Using the squaring operator $Sy := y \op y$ and the left and right multiplication operators $L_y x := y \op x$ and $R_y x = x \op y$, this law can be written as
$$ L_y R_y L_{Sy} x = x.$$
In particular, this gives a way to construct these magmas:

\begin{lemma}[Construction of 1323 magmas]\label{1323-construct}\leanok\lean{Eq1323.eq1323_if_conditions}  Suppose that $M$ is a magma such that
  \begin{equation}\label{lr}
    R_{Sy} L_{Sy} = 1
   \end{equation}
and
\begin{equation}\label{lr-simp}
  L_{y} R_{y} = R_{Sy}
\end{equation}
hold.  Then the magma obeys 1323.
\end{lemma}

\begin{proof} Trivial.
\end{proof}

 So now we would like to construct magmas satisfying \Cref{lr} and \Cref{lr-simp}.  We need some bijections:

 \begin{lemma}[Bijections]\label{bij}\leanok\lean{Eq1323.ϕ}
  Let $G$ be a countably infinite abelian torsion group of exponent $2$.  Then there exists a bijection $\phi_a: G \to \Q^\times$ for each $a \in G \backslash \{0\}$ such that
 $\phi_a(0) = 1$ and $\phi_a(a+b) = -\phi_a(b)$ for all $b \in G$, so in particular $\phi_a(a)=-1$.
 \end{lemma}

 \begin{proof}
  Such a bijection can be easily constructed from the axiom of choice and a greedy algorithm, defining $\phi_a$ one pair $\{b,a+b\}$ at a time.
 \end{proof}


 \begin{lemma}[Building a magma]\label{build-magma}\leanok\lean{Eq1323.op}\lean{Eq1323.op_RSy_LSy_eq_Id}\lean{Eq1323.op_Ly_Ry_eq_LSy}  Let $G$ be a countably infinite abelian torsion group of exponent $2$, and let $\phi_a$ be as in the previous lemma.  Let $N$ be the set of pairs $(x,a)$ with $x \in \Q^\times$ and $a \in G \backslash \{0\}$, and let $M = G \uplus N$ be the disjoint union of $G$ and $N$.  Suppose that we have an operation $\op : M \times M \to M$ obeying the following axioms:
\begin{itemize}
\item (i) We have
\begin{equation}\label{op-0}
  a \op b = a+b
\end{equation}
for $a,b \in G$.
\item (ii) We have
 \begin{equation}\label{op-1}
  (x,a) \op b = (\phi_a(b) x, a)
 \end{equation}
 \begin{equation}\label{op-2}
   b \op (x,a) = (x / \phi_a(b), a)
 \end{equation}
 and
 \begin{equation}\label{op-3}
   (\phi_a(b) x,a) \op (x,a) = a+b
 \end{equation}
 for $x \in \Q^\times$, $b \in G$, and $a \in G \backslash \{0\}$.
\item (iii) If $a,b,0 \in G$ are distinct and
$(x,a) \op (y,b) = (z,c)$ for some $x,y,z \in \Q^\times$ and $c \in G$, then $(y,b) \op (z,c) = (\phi_a(b) x,a)$.
\end{itemize}
Then \Cref{lr}, \Cref{lr-simp} and hence \Cref{1323} holds.
\end{lemma}

\begin{proof}\uses{bij}
  With these rules, $0$ is a unit, and the squaring operator is given by $Sa = 0$ and $S(x,a) = a$, so the set of squares is $G$.  If $a \in G$ is a square number, we have
  $$ L_a b = R_a b = a+b$$
  and
  $$ L_a (x,b) = (x/\phi_b(a), b); \quad R_a (x,b) = (x \phi_b(a), b)$$
  so \Cref{lr} is satisfied.

Now we verify \Cref{lr-simp}.
If $y$ is a square, then the claim already follows from \Cref{lr} since $Sy=0$ is a unit.  Otherwise, for $y \in \Q^\times$ and $b \in G \backslash \{0\}$, we have to show that
$$ L_{(y,b)} R_{(y,b)} a = R_b a = a+b$$
for $a \in G$ and
$$ L_{(y,b)} R_{(y,b)} (x,a) = R_b (x,a) = (\phi_a(b) x, a) $$
for $x \in \Q^\times$ and $a \in G \backslash \{0\}$.  In the first case, we have from \Cref{op-2} that
$$ R_{(y,b)} a = (y/\phi_a(b), b)$$
and then from \Cref{op-3} we have
$$L_{(y,b)} R_{(y,b)} a = a+b$$
as required.  In the second case, if $a=b$, then by the bijective nature of $\phi_a$, we can write $x = \phi_a(c) y$ for some $c \in G$, then from \Cref{op-3} we have
$$ R_{(y,b)} (x,a) = a+c$$
and then by \Cref{op-1}
$$ L_{(y,b)} R_{(y,b)} (x,a) = (\phi_a(a+c) y, a)$$
but from construction we have $\phi_a(a+c)=-\phi_a(c) = \phi_a(b)\phi_a(c)$ and hence $\phi_a(a+c) y = \phi_a(b) x$ as required.

It remains to handle the case when $a,b$ are distinct elements of $G \backslash \{0\}$.  But this follows from (iii).
\end{proof}

\begin{definition}[Partial solution]\label{partial-1323} A \emph{partial solution} is a finite family ${\mathcal F}$ of tuples $(x,y,z,a,b,c) \in (\Q^\times)^3 \times G^6$ with $a,b,c,0$ distinct, such that the tuples
$(\phi_a(b)^n x,a, \phi_b(c)^n y,b)$, $(\phi_b(c)^n y,b,\phi_c(a)^n z,c)$, $(\phi_c(a)^n z,c,\phi_a(b)^{n+1} x, a)$ for $(x,y,z,a,b,c) \in {\mathcal F}$ and $n \geq 0$ are all distinct.
\end{definition}

\begin{lemma}[Soundness]\label{partial-1323-sound}\uses{partial-1323}  Let ${\mathcal F}$ be a partial solution.  Then if one defines a partial operation $\op$ on $M$ by requiring axioms (i), (ii), imposing the additional operations
$$ (\phi_a(b)^n x,a) \op (\phi_b(c)^n y,b) = (\phi_c(a)^n z,c)$$
$$ (\phi_b(c)^n y,b) \op (\phi_c(a)^n z,c) = (\phi_a(b)^{n+1} x, a)$$
$$ (\phi_c(a)^n z,c) \op (\phi_a(b)^{n+1} x, a) = (\phi_b(c)^{n+1} y,b)$$
for all $n \geq 0$ and $(x,y,z,a,b,c) \in {\mathcal F}$, and no other operations, then this is a well-defined partial operation that obeys axiom (iii) whenever it is defined.
\end{lemma}

\begin{proof} Routine.
\end{proof}

\begin{lemma}[Greedy extension]\label{greedy-1323}\uses{partial-1323-sound}\leanok\lean{Eq1323.extend}  If $\op$ is defined by a partial solution, and $(x,a) \op (y,b)$ is undefined for some $x,y \in \Q^\times$ and distinct $a,b \in G \backslash \{0\}$, then it is possible to extend the partial solution so that $(x,a) \op (y,b)$ is now defined.
\end{lemma}

\begin{proof} We select a $c \in G \backslash \{0\}$ that has not previously been used by the partial solution, let $z \in \Q^\times$ be arbitrary, (e.g., one could take $z=1$ if desired), and add $(x,y,z,a,b,c)$ to ${\mathcal F}$, thus we now also assign
$$ (\phi_a(b)^n x,a) \op (\phi_b(c)^n y,b) = (\phi_c(a)^n z,c)$$
$$ (\phi_b(c)^n y,b) \op (\phi_c(a)^n z,c) = (\phi_a(b)^{n+1} x, a)$$
$$ (\phi_c(a)^n z,c) \op (\phi_a(b)^{n+1} x, a) = (\phi_b(c)^{n+1} y,b)$$
for all natural numbers $n$.  As the $a,b,c$ are distinct, the $\phi_a(b), \phi_b(c), \phi_c(a)$ are not equal to $\pm 1$, and the members of this infinite sequence do not collide with each other.  The second and third equations in this family cannot collide with previous assignments because $c$ is novel.  If we arrange matters so that $\phi_b(c)$ involves primes in the numerator or denominator that do not appear in any previous $\phi_a(b), \phi_b(c), \phi_c(a), x, y, z$ used by the greedy algorithm, or the current $x,y,z$, then we also see that the first infinite sequence does not collide with any previously assigned value of $\op$ either.
\end{proof}

\begin{corollary}[Iterated greedy extension]\label{greedy-iterate}\leanok\lean{Eq1323.exists_complete_function}  Every partial solution can be extended to a complete solution that obeys \Cref{lr} and \Cref{lr-simp}, and hence 1323.
\end{corollary}

\begin{proof}\uses{greedy-1323, build-magma, partial-1323-sound} Apply the usual greedy algorithm.
\end{proof}

\begin{corollary}[1323 does not imply 2744]\label{1323-refute-2744}\leanok\lean{Eq1323.Equation1323_not_implies_Equation2744}  There exists a 1323 magma which does not obey the 2744 equation $R_y L_{Sy} L_y x = x$.
\end{corollary}

\begin{proof}\uses{greedy-iterate} It suffices to produce a partial solution in which $L_y$ is not injective. Pick distinct $a, b, b', c$ with $\phi_a(b)$, $\phi_a(b')$ having no prime factors in common in the numerator or denominator.  We take the partial solution consisting of $(1,1,1,a,b,c)$ and $(1,1,1,a,b',c)$, that is to say we impose the conditions
$$ (\phi_a(b)^n,a) \op (\phi_b(c)^n,b) = (\phi_c(a)^n,c)$$
$$ (\phi_b(c)^n,b) \op (\phi_c(a)^n,c) = (\phi_a(b)^{n+1}, a)$$
$$ (\phi_c(a)^n,c) \op (\phi_a(b)^{n+1}, a) = (\phi_b(c)^{n+1},b)$$
and
$$ (\phi_a(b)^n,a) \op (\phi_{b'}(c)^n,b') = (\phi_c(a)^n,c)$$
$$ (\phi_{b'}(c)^n,b') \op (\phi_c(a)^n,c) = (\phi_a(b')^{n+1}, a)$$
$$ (\phi_c(a)^n,c) \op (\phi_{a}(b')^{n+1}, a) = (\phi_b(c)^{n+1},b').$$
One can check that no collisions occur; on the other hand, as $L_{(1,a)} (1,b) = L_{(1,a)} (1,b') = (1,c)$, we already have a violation of left injectivity.  Completing this seed to a full magma using \Cref{greedy-iterate}, we obtain the claim.
\end{proof}
